\documentclass[12pt,a4paper,utf8]{ctexart}
\usepackage{ctex,amsmath,amssymb,subfig,cite,graphicx,diagbox,fontspec,fancyhdr,geometry}
\usepackage[ntheorem]{empheq}
\usepackage{enumitem,fullpage,cleveref,cellspace,listings,color,framed}
\definecolor{gray}{rgb}{0.5,0.5,0.5}
\definecolor{dkgreen}{rgb}{.068,.578,.068}
\definecolor{dkpurple}{rgb}{.320,.064,.680}

%set Fortran styles
\lstset{
    frameround=tftf,
    language=Fortran,
    keywords={SELECT,PROGRAM,PRINT,STOP,END,WRITE,INTEGER,REAL,COMPLEX,CHARACTER,LOGICAL,READ,FORMAT,IMPLICIT,PARAMETER,DATA,EQUIVALENCE,TYPE,PAUSE,CONTINUE,CYCLE,EXIT,IF,SELECT,DO,ALLOCATE,DEALLOCATE,WHERE,FORALL,SUBROUTIHNE,CALL,RETURN,FUNCTION,COMMON,BLOCK DATA,SAVE,INTERFACE,CONTAIN,MODULE,USE,PUBLIC,PRIVATE,ENTRY,OPEN,INQUIRE,CLOSE,NAMELIST,POINTER,NULLFY,REWIND,BACKSPACE,ENDFILE
    },
    basicstyle=\small\ttfamily,
    numbers=left,
    numberstyle=\small,
    keywordstyle=\color{blue}\bfseries,
    commentstyle=\color{dkgreen},
    stringstyle=\color{dkpurple},
    backgroundcolor=\color{white},
    tabsize=2,
    showspaces=false,
    showstringspaces=false,
    breaklines=true,
    frame=trBL,
}
\CTEXsetup[format+={\raggedright}]{section}
\setlength{\parindent}{2em}
\geometry{
    textwidth=138mm,
    textheight=215mm,
    left=27mm,
    right=27mm,
    top=25.4mm,
    bottom=25.4mm,
    headheight=2.17cm,
    headsep=4mm,
    footskip=12mm,
    heightrounded,
}
\pagestyle{fancy}
\lhead{\textsl{2021秋-计算物理A}}
\chead{}
\rhead{\textsl{PB19020634-于浩然}}
\lfoot{}
\cfoot{\thepage}
\rfoot{}

\begin{document}
\begin{center}
    {\LARGE\textbf{计算物理作业四}}\\
    \textrm{于浩然}~~~~~~\textrm{PB19020634}~~~~~~\textrm{2021.10.17}
\end{center}

\section{作业题目}

设pdf函数满足关系式:
\begin{equation}
    p'(x)=a \delta (x)+b \exp (-cx), \qquad x \in [-1,1]
\end{equation}
讨论该函数性质并给出抽样方法.

\section{函数性质讨论}

\subsection{函数形式的确定}

pdf函数的导数中含有$\delta(x)$,故pdf函数在$x=0$处有阶跃.我们引进Heaviside阶梯函数:
\begin{equation}
    \Theta (s) = 
    \begin{cases}
        1,\quad s>0 \\
        0,\quad s<0 
    \end{cases}
\end{equation}

此函数具有如下性质:
\begin{equation}
    \frac{ \textrm{d}\Theta (s)}{ \textrm{d}s} = \delta (s)
\end{equation}

于是,我们不妨设$p'(x)$原函数的第一项为
\begin{equation}
    p_1(x) = a \Theta (x)
\end{equation}

对$p'(x)$的第二项求不定积分:
\begin{equation}
    \int b \exp(-cx) \textrm{d}x= - \frac{b}{c} e^{-cx} + C
\end{equation}

为了讨论方便,我们尽可能使$p(x)$的形式简单.不妨取$C=0$,得到pdf函数的第二项:
\begin{equation}
    p_2(x) = - \frac{b}{c} e^{-cx}
\end{equation}

至此我们已经得到了pdf函数的一个简单形式:
\begin{eqnarray}
    p(x) &=& p_1(x) + p_2(x) \nonumber \\
         &=& a \Theta (x) - \frac{b}{c} e^{-cx} \nonumber \\
         &=& 
         \begin{cases}
             a - \frac{b}{c} e^{-cx},\quad x \in [0,1] \\
             - \frac{b}{c} e^{-cx},\qquad x \in [-1,0]
         \end{cases}
\end{eqnarray}

\subsection{函数最大值的讨论}

由于后面可能需要用到舍选抽样法,考虑其抽样效率,需要对pdf函数的最大值进行讨论.由于pdf函数不能为负,$a,b,c$参数将受到一定限制:
\begin{equation}
    \begin{cases}
        a - \frac{b}{c}e^{-cx} > 0,\quad x \in [0,1] \\
        - \frac{b}{c} e^{-cx} > 0,\qquad x \in [-1,0]
    \end{cases}
\end{equation}

易得$-b/c > 0$.下面分类讨论:
\begin{itemize}
    \item $c > 0 \Rightarrow b < 0$

        这时$e^{-cx}$为递减函数,在$-1$处取最大值$e^{c}$.这时函数的最大值仍决定于$a$的大小,为了进行讨论,我们将函数分为两段分别求最大值:
        \begin{equation}
            \begin{cases}
                \max_{x \in [0,1]}p(x) = a - b/c \\
                \max_{x \in [-1,0]}p(x) = - (b/c)e^{c}
            \end{cases}
        \end{equation}
    \item $c < 0 \Rightarrow b > 0$

        这时$e^{-cx}$为递增函数,在$1$处取最大值$e^{-c}$.有
        \begin{equation}
            a > \frac{b}{c}e^{-c} 
        \end{equation}
        $a$仍可能为负值,故我们考虑最大值时仍分段考虑:
        \begin{equation}
            \begin{cases}
                \max _{x \in [0,1]}p(x) = a - (b/c)e^{-c} \\
                \max _{x \in [-1,0]}p(x) = -b/c
            \end{cases}
        \end{equation}
\end{itemize}

\subsection{求反函数的可能性}

若要使用直接抽样法,则须考虑pdf函数的累积函数是否容易求反函数.对我们给出的pdf函数形式(7),求其累积函数:
\begin{eqnarray}
    \eta(x) &=& \int _{-1} ^{x} p(\tau) \textrm{d}\tau \nonumber \\
           &=& 
           \begin{cases}
               (b/c^2)(e^{-cx}-e^c),~~~~~~~~~ x \in [-1,0] \\
               (b/c^2)(e^{-cx}-e^c) + ax,~~ x \in [0,1] 
           \end{cases} \nonumber \\
           &=& (b/c^2)(e^{-cx}-e^c) + ax \Theta(x)
\end{eqnarray}

对于这样的累积函数,求反函数十分困难,即使分段后$[0,1]$上也不能解析求出$x(\eta)$,故我们在后面不考虑直接抽样法.
\section{抽样方法}

\subsection{抽样公式推导}

在这里我们直接考虑 \textbf{阶段函数舍选法}.分别取$[0,1]$和$[-1,0]$上的最大值
\begin{equation}
    M_1 = a - \frac{b}{c}e^{|c|},\qquad M_2 = - \frac{b}{c}e^{|c|}
\end{equation}

设分段阶梯比较函数
\begin{equation}
    F(x) = 
    \begin{cases}
        M_1,\quad x \in [0,1] \\
        M_2,\quad x \in [-1,0]
    \end{cases}
\end{equation}

根据舍选法的一般形式:
\begin{enumerate}
    \item[(1)] 产生一对$[0,1]$区间中均匀分布的随机抽样值$(\xi_1,
        \xi_2)$,可得抽样表示式:
    \begin{equation}
        \xi_1 = \int _a ^{\xi_x}F(x) \textrm{d}x\big / \int_{-1} ^1 F(x) \textrm{d}x,
        \qquad \xi _y = \xi_2 F(\xi_x)
    \end{equation}

    分段表示如下:
    \begin{equation}
        \xi_1= \left\{  
            \begin{aligned}
                &\frac{(\xi_x + 1)M_2}{M_1 + M_2},\qquad \xi_x \in [-1,0] \\
                &\frac{M_2 + \xi_x M_1}{M_1+M_2}, \qquad \xi_x \in [0,1]
            \end{aligned}
            \right .
    \end{equation}
    \begin{equation}
        \xi_2= \left\{
            \begin{aligned}
                &\xi_y / M_2 , \qquad \xi_x \in [-1,0] \\   
                &\xi_y / M_1 , \qquad \xi_x \in [0,1]
            \end{aligned} \right.
    \end{equation}

\item[(2)] 判断条件 $\xi_y \leq p(\xi_x)$是否成立:

    \begin{itemize}
        \item $\xi_x \in [-1,0]$时
            \begin{equation}
                \xi_x = (1+ \frac{M_1}{M_2}) \xi_1 -1 < 0 \Rightarrow
                \xi_1 < \frac{M_2}{M_1+M_2}
            \end{equation}

            判断条件为
            \begin{equation}
                M_2 \xi_2 < p((1+ \frac{M_1}{M_2})\xi_1 - 1) 
            \end{equation}
            时,取 $\xi_x$.

        \item $\xi _x \in [0,1]$时
            \begin{equation}
                \xi_x = \frac{(M_1+M_2)\xi_1 - M_2}{M_1} > 0 \Rightarrow \xi_1 >
                \frac{M_2}{M_1+M_2}
            \end{equation}

            判断条件为
            \begin{equation}
                M_1 \xi_2 < p\left( (1+ \frac{M_2}{M_1})\xi_1 - \frac{M_2}{M_1} \right)
            \end{equation}
            时,取$\xi_x$.
    \end{itemize}
            若判断条件不成立,则舍.
\end{enumerate}

\subsection{抽样效率讨论}

对于舍选法抽样,其抽样效率(即有效选取的点数占比)即为$p$与$F$的面积比.代入本题目中参数可得抽样效率:
\begin{equation}
    \frac{\int _{-1}^{1}p(x) \textrm{d}x}{\int _{-1} ^1 F(x) \textrm{d}x} 
    = \frac{a + \int _{-1} ^1 (- \frac{b}{c})e^{-cx}}{M_1+M_2}
    = \frac{a+b/c^2(e^c-e^{-c})}{a - 2(b/c) e^{|c|}}
\end{equation}
\section{结论}

本题中我们讨论了带有阶跃的pdf函数的性质与抽样方法.对于比较复杂的情况,直接抽样法中的反函数通常很难求解,这更体现了舍选抽样法“万金油”的优越性.对于简单分布(密度分布函数$p(x)$定义在有限区域且有界),可通过直接采取分段式常数函数作为比较函数的方式,便捷地得到抽样效率可观的抽样方法.

本题中变参数过多,可以说是$a,b,c$的取值决定了我们的抽样方法好坏,为防止得到片面性结论.决定不自行确定数值来进行实验.
\color{red} 注意归一性没有讨论到.
\end{document}
