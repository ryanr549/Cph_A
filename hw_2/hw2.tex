\documentclass[qpt,a4paper,utf8]{ctexart}
\usepackage{ctex,amsmath,amssymb,subfig,cite,graphicx,diagbox,fontspec,fancyhdr,geometry}
\usepackage[ntheorem]{empheq}
\usepackage{enumitem,fullpage,cleveref,cellspace,listings,color,framed,indentfirst}
\definecolor{gray}{rgb}{0.5,0.5,0.5}
\definecolor{dkgreen}{rgb}{.068,.578,.068}
\definecolor{dkpurple}{rgb}{.320,.064,.680}

%set Fortran styles
\lstset{
    frameround=tftf,
    language=Fortran,
    keywords={SELECT,PROGRAM,PRINT,STOP,END,WRITE,INTEGER,REAL,COMPLEX,CHARACTER,LOGICAL,READ,FORMAT,IMPLICIT,PARAMETER,DATA,EQUIVALENCE,TYPE,PAUSE,CONTINUE,CYCLE,EXIT,IF,SELECT,DO,ALLOCATE,DEALLOCATE,WHERE,FORALL,SUBROUTIHNE,CALL,RETURN,FUNCTION,COMMON,BLOCK DATA,SAVE,INTERFACE,CONTAIN,MODULE,USE,PUBLIC,PRIVATE,ENTRY,OPEN,INQUIRE,CLOSE,NAMELIST,POINTER,NULLFY,REWIND,BACKSPACE,ENDFILE
    },
    basicstyle=\small\ttfamily,
    numbers=left,
    numberstyle=\small,
    keywordstyle=\color{blue}\bfseries,
    commentstyle=\color{dkgreen},
    stringstyle=\color{dkpurple},
    backgroundcolor=\color{white},
    tabsize=2,
    showspaces=false,
    showstringspaces=false,
    breaklines=true
}
\CTEXsetup[format+={\raggedright}]{section}
\setlength{\parindent}{2em}
\geometry{
    textwidth=138mm,
    textheight=215mm,
    left=27mm,
    right=27mm,
    top=25.4mm,
    bottom=25.4mm,
    headheight=2.17cm,
    headsep=4mm,
    footskip=12mm,
    heightrounded,
}
\pagestyle{fancy}
\lhead{\textsl{2021秋-计算物理A}}
\chead{}
\rhead{\textsl{PB19020634-于浩然}}
\lfoot{}
\cfoot{\thepage}
\rfoot{}

\begin{document}
\begin{center}
    {\LARGE\textbf{计算物理作业二}}\\
    \textrm{于浩然}~~~~~~\textrm{PB19020634}~~~~~~\textrm{2021.10.07}
\end{center}
\section{作业题目}
用16807产生器测试随机数序列中满足关系
$X_{n-1}>X_{n+1}>X_n$的比重。讨论Fibonacci延迟测试器中出现这种关系的比重。
\section{算法简介}

线性同余法是最简单的生成均匀随机数的方法,其优势在于速度快,但是这样产生的随机数在序列相关性上较差。例如,在32位计算机上
$m$可能取的最大值 $~2^{31}$,在3维空间中连续使用随机数的话,点将分布在空间中最多达
$[3!(2^{31})]^{1/3}=2344$个平面上。因此,当我们只关注体积中一小部分的时候,平面的规则离散性对计算结果有很大影响。

\textbf{Fibonacci延迟产生器}是另一种类型的随机数产生器,其中有些特例可以用来消除同余法中的关联问题。其思想是用序列中的两个整数进行操作得到后续的整数(名称源于Fibonacci数列$1,2,3,5,8,13,21,34, \cdots$,其中一个整数为前两个数之和),表达式为:
\begin{equation}
    I_n=I_{n-p} \otimes I_{n-q} \mathrm{mod} m
\end{equation}

操作符 $\otimes$可以是加、减、乘、除、XOR(与或)。整数对
$[p,q]$表示延迟$(p>q)$,取值一般通过统计检验后确定。Fibonacci延迟产生器与LCG相比其周期非常长,在32位机上的最大周期为
$(2^p-1)2^{31}$.

一种Fibonacci延迟器的例子便是
\textbf{Marsaglia}产生器(组合产生器,由两个不同的随机数产生器生成另一个随机数序列)中第一个随机数产生器,即一种减法操作Fibonacci延迟产生器(\textbf{后文简称为减法生成器}):
\begin{equation}
    x_n=\left\{
        \begin{aligned}
        & x_{n-p}-x_{n-q} \quad & ,if \geq 0 \\   
        & x_{n-p}-x_{n-q}+1 & ,otherwise
        \end{aligned}
        \right.
\end{equation}

若用(1)来表达,等价于 $0 \geq I_n < 1$,且 $m=1$.其中 $[p,q]$整数对选取
$[97,33]$,因此该算法要求存储所有前面的97个随机数值。

为方便起见,我们将要检验的条件列出:
\begin{equation}
X_{n-1}>X_{n+1}>X_n 
\end{equation}

\section{编程实现}

用Fortran90进行编程。共调用3个子程序,各自功能与源代码如下:

\begin{itemize}
    \item \texttt{SUBROUTIHNE Minus(N)}\\
\textbf{减法产生器}需要97个种子来生成随机数,考虑到之前编写过Schrage方法LCG的子程序,直接调用该子程序,用种子
\texttt{In(0) = m - 1}生成
$10^2$个随机数作为种子来生成随机数。传入参数 \texttt{N}决定随机数总数目为
$10^N$,并将得到的随机数序列存入文件 \texttt{minrand.out}.

\begin{framed}
\begin{lstlisting}[frame=trBL]
SUBROUTINE Minus(N) !用Marsaglia生成器中的“减法生成器”生成随机数序列
    INTEGER(KIND=4) :: p = 97, q = 33, N, i
    REAL(KIND=8) :: x(10**N), z(100) 
    CALL Schrage(2) !用16807生成器生成10^2个随机数作为“减法生成器”的种子
    OPEN (99, file='lcgrand.out')
    READ (99, *) z !首先将种子读取到数组z中
    CLOSE (99)
    DO i = 1, 97 !由种子生成前97个随机数
        x(i) = z(97 + i - p) - z(97 + i - q) !x的第一个数据由z(1)和z(65)生成
        IF(x(i) < 0) THEN
            x(i) = x(i) + real(1)
        END IF
    END DO
    DO i = 98, 10**N  !由生成的前97个随机数生成总数目为10^N的随机数序列
        x(i) = x(i - p) - x(i - q)
        IF(x(i) < 0) THEN
            x(i) = x(i) + real(1)
        END IF
    END DO
    OPEN (99, file='minrand.out') !将随机数按行存入文件
    DO i = 1, 10**N
        WRITE (99, *) x(i)
    END DO
    CLOSE (99)

END SUBROUTINE Minus
\end{lstlisting}
\end{framed}

\item \texttt{SUBROUTINE Schrage(P)}\\
即为前一题所用的16807产生器,产生 $10^P$个随机数写入文件 \texttt{lcgrand.out}.

\begin{framed}
\begin{lstlisting}[frame=trBL]
SUBROUTINE Schrage(P) !Schrage随机数生成器子程序
    IMPLICIT NONE
    INTEGER :: N = 1, P
    INTEGER :: m = 2147483647, a = 16807, q = 127773, r = 2836, In(10**P)
    REAL(KIND=8) z(10**P)
    In(1) = m - 1
    z(1) = REAL(In(1))/m
    DO N = 1, 10**P - 1
        In(N + 1) = a*MOD(In(N), q) - r*INT(In(N)/q)
        IF (In(N + 1) < 0) THEN !若值小于零,按Schrage方法加m
        In(N + 1) = In(N + 1) + m
        END IF
        z(N + 1) = REAL(In(N + 1))/m !得到第N+1个随机数
    END DO
    OPEN (1, file='lcgrand.out') !每次运行子程序将覆盖随机数
    DO N = 1, 10**P !将随机数按行存入文件
        WRITE (1, *) z(N)
    END DO
    CLOSE (1)
END SUBROUTINE Schrage
\end{lstlisting}
\end{framed}
    
\item \texttt{SUBROUTINE Test(N, Filename)}\\
传入随机数序列大小 $10^N$,以及通过前面两个程序生成的随机数文件名
\texttt{Filename},读取序列加载到实型数组
\texttt{z(10**N)}中,随后遍历每一个元素,验证是否满足我们验证的关系
(3),最后得到满足条件的元素数目 \texttt{Number},与总数
$10^N$作商求出比值,写入文件 \texttt{ratio.out}.
\begin{framed}
\begin{lstlisting}[frame=trBL]
SUBROUTINE Test(N, Filename) !检验随机数序列中出现关联比重的子程序
    INTEGER(KIND=4) :: Number = 0, i = 1
    CHARACTER(LEN=11) :: Filename
    REAL(KIND=8) :: x(10**N), r !定义要检验的随机数序列
    OPEN (11, file=Filename ) !打开相应文件读取随机数
    READ (11,*) x
    CLOSE (11)
    IF (x(10**N) > x(2) .AND. x(2) > x(1)) THEN !用x(10**N)代替x(0)
        Number = Number + 1 !第1个随机数和最后一个随机数中最多只能有一个满足我们讨论的关系
    ELSE IF (x(10**N - 1) > x(1) .AND. x(1) > x(10**N)) THEN !用x(1)代替x(10**N + 1)
        Number = Number + 1
    END IF
    DO i = 2, 10**N - 1
        IF(x(i - 1) > x(i + 1) .AND. x(i + 1) > x(i)) THEN
            Number = Number + 1 !每当检验满足关系,累加Number
        END IF
    END DO
    r = real(Number) / 10**N !计算出现关联性的比重
    OPEN (1, ACCESS='append', file='ratio.out')
    WRITE (1, *) r
    CLOSE(1)
END SUBROUTINE Test
\end{lstlisting}
\end{framed}
\end{itemize}
\newpage
在主程序中使用
\texttt{DO}循环结构,调用上述子程序生成不同大小的随机数序列并对其进行性质(3)的检验.

\begin{framed}
\begin{lstlisting}[frame=trBL]
PROGRAM MAIN
    IMPLICIT NONE
    INTEGER :: N = 1 !用16807生成器生成100个随机数作为“减法生成器”的种子
    !PRINT *, "Input required quantity of random numbers(10^N)"
    !READ (*,*) N
    DO N = 2 , 7
        CALL Minus(N)
        CALL Schrage(N)
        CALL Test(N, 'minrand.out') !分别传入两个文件名作为实参来对其进行测试
        CALL Test(N, 'lcgrand.out')
    END DO
END PROGRAM MAIN
\end{lstlisting}
\end{framed}

\section{计算结果}
\noindent \textbf{两种产生器出现关系的比重:}

        分别读取
        \texttt{ratio.out}中偶数位次的数值(即16807产生器得到的数值)以及奇数位次的数值(即减数产生器/Fibonacci延迟产生器得到的数值),整理列表展示如下:

\begin{table*}[h]
\centering
\begin{tabular}{|l|l|l|l|l|l|l|}
\hline
\diagbox{RNG type}{N}              & $10^2$  & $10^3$  & $10^4$  & $10^5$  & $10^6$  & $10^7$  \\ \hline
16807 RNG     & 0.30000 & 0.38300 & 0.41140 & 0.24127 & 0.19423 & 0.18636 \\ \hline
Fibonacci RNG & 0.15000 & 0.22300 & 0.24560 & 0.07597 & 0.02761 & 0.01977 \\ \hline
\end{tabular}
\caption{两种产生器出现关系的比重比较}
\end{table*}

从表中很容易看出:
\begin{itemize}
    \item
        Fibonacci RNG所产生的随机数序列中出现关系(3)的比重明显小于16807 RNG.
    \item 随着生成随机数数目增多,出现关系的比重都会下降:但当数目从 $10^4$变为
        $10^5$时,Fibonacci
        RNG出现关系的比重显著下降了一个量级,并且在数目继续增大时其值与16807
        RNG的差距进一步增大.
\end{itemize}

由此我们得出结论:Fibonacci延迟产生器在序列相关性上的性质优于传统的16807线性同余产生器,当生成随机数序列较大时优势更加明显.
\newpage
\section{总结}
本作业中我们编写了“减法生成器”——一种Fibonacci延迟器的子程序,可用于生成序列相关性比线性同余产生器更弱的随机数序列;对两种不同产生器我们分别测试了性质,并比较出了性质的优劣,这可以启发我们确定选取随机数产生器的标准.
\color{red} 本题有理论概率值,讨论结果是Fibonacci延迟器效果更差.
\end{document}
