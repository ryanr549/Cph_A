\documentclass[12pt,a4paper,utf8]{ctexart}
\usepackage{ctex,amsmath,amssymb,subfig,cite,graphicx,diagbox,fontspec,fancyhdr,geometry}
\usepackage[ntheorem]{empheq}
\usepackage{enumitem,fullpage,cleveref,cellspace,listings,color,framed}
\definecolor{gray}{rgb}{0.5,0.5,0.5}
\definecolor{dkgreen}{rgb}{.068,.578,.068}
\definecolor{dkpurple}{rgb}{.320,.064,.680}

%set Fortran styles
\lstset{
    frameround=tftf,
    language=Fortran,
    keywords={SELECT,PROGRAM,PRINT,STOP,END,WRITE,INTEGER,REAL,COMPLEX,CHARACTER,LOGICAL,READ,FORMAT,IMPLICIT,PARAMETER,DATA,EQUIVALENCE,TYPE,PAUSE,CONTINUE,CYCLE,EXIT,IF,SELECT,DO,ALLOCATE,DEALLOCATE,WHERE,FORALL,SUBROUTIHNE,CALL,RETURN,FUNCTION,COMMON,BLOCK DATA,SAVE,INTERFACE,CONTAIN,MODULE,USE,PUBLIC,PRIVATE,ENTRY,OPEN,INQUIRE,CLOSE,NAMELIST,POINTER,NULLFY,REWIND,BACKSPACE,ENDFILE
    },
    basicstyle=\small\ttfamily,
    numbers=left,
    numberstyle=\small,
    keywordstyle=\color{blue}\bfseries,
    commentstyle=\color{dkgreen},
    stringstyle=\color{dkpurple},
    backgroundcolor=\color{white},
    tabsize=2,
    showspaces=false,
    showstringspaces=false,
    breaklines=true,
    frame=trBL,
}
\CTEXsetup[format+={\raggedright}]{section}
\setlength{\parindent}{2em}
\geometry{
    textwidth=138mm,
    textheight=215mm,
    left=27mm,
    right=27mm,
    top=25.4mm,
    bottom=25.4mm,
    headheight=2.17cm,
    headsep=4mm,
    footskip=12mm,
    heightrounded,
}
\pagestyle{fancy}
\lhead{\textsl{2021秋-计算物理A}}
\chead{}
\rhead{\textsl{PB19020634-于浩然}}
\lfoot{}
\cfoot{\thepage}
\rfoot{}

\begin{document}
\begin{center}
    {\LARGE\textbf{计算物理作业八}}\\
    \textrm{于浩然}~~~~~~\textrm{PB19020634}~~~~~~\textrm{2021.10.20}
\end{center}

\section{作业题目}

用Monte Carlo方法计算如下定积分,并讨论有效数字位数.
\begin{equation}
    \int _0 ^5 \textrm{d}x \sqrt{x+ 2 \sqrt{x}};
\end{equation}
\begin{equation}
    \int _0 ^{7/10} \textrm{d}x \int _0 ^{4/7} \textrm{d}y \int _0 ^{9/10}
    \textrm{d}z \int _0 ^2 \textrm{d} u \int _0 ^{13/11} \textrm{d} v (5 - x^2 +y^2 -
    z^2 + u^3 - v^3)
\end{equation}

\section{算法简介}

\subsection{平均值法计算一维积分}

根据积分的平均值定理:
\begin{equation}
    \int _{a} ^{b} f(x) \textrm{d}x = (b - a)\langle f \rangle
\end{equation}

而平均值又可从下式得到:
\begin{equation}
    \langle f \rangle \approxeq \frac{1}{N} \sum_{i=1}^{N}f(x_i)
\end{equation}

故有
\begin{equation}
    \int _{a} ^{b} f(x)	\textrm{d}x \approxeq \frac{(b-a)}{N} \sum_{i=1}^{N}
    f(x_i)
\end{equation}

此即Monte Carlo方法计算一维定积分.

\subsection{多重定积分}

Monte Carlo方法真正的威力在于应用于多重积分.将(5)式推广为:
\begin{equation}
    \int _{a_1} ^{b_1} \textrm{d} x_1	\int _{a_2} ^{b_2} \textrm{d}x_2	
    \cdots \int _{a_n} ^{b_n} \textrm{d}x_n	f(x_1,x_2,\cdots,x_n)
    = \frac{1}{N}\left[ \prod_{j=1}^{n} (b_j - a_j)\right]
    \sum_{i=1}^{N}f(x_1,x_2,\cdots,x_n)
\end{equation}
其中对每个坐标的抽样值在相应区间范围内均匀抽取.

\section{编程实现}

用FORTRAN90进行编程,将积分程序写在一个模块
\texttt{integrate}中.其中包含:
\begin{itemize}
    \item \texttt{f(x)} 、\texttt{g(x, y, z, u, v)}
        
        分别用于表示两个被积函数.
    \item \texttt{SUBROUTIHNE integrate\_1(n)}

        此子程序为1维Monte Carlo积分计算程序.在这里我们调用16807生成器程序 
        \texttt{Schrage(P, z0, filename)},生成$10^n$个均匀分布随机数.
        \texttt{ave}为计算前
        \texttt{i}个数的均值,每当增加一个数时更新均值如下:
        \begin{equation}
            \bar{x}_N = \bar{x}_{N-1} + \frac{x_N - \bar{x}_{N-1}}{N}
        \end{equation}
        其中$\bar{x}_N = \frac{1}{N}
        \sum_{i=1}^{N}x_i$.在程序中即23行所示,我们采用这种方法以避免求和时
        发生数据溢出.
    \item \texttt{SUBROUTIHNE integrate\_1(n)}
        此子程序思路与前面基本相同,但使用了$10^n \times 5$维的数组,
        5列分别用于存放
        \texttt{x,y,z,u,v},具体实现见38-42行的循环结构.数组的5个维度出自同一个随机数序列,我们
        曾在第一道作业题中论证过这些子序列可认为是互不相关的,故
        这样做与分别产生5个随机数序列无异.
\begin{framed}
\begin{lstlisting}[language=Fortran]
MODULE integrate !聚合两个积分程序的模块
    IMPLICIT NONE
    CONTAINS
        FUNCTION f(x) !定义被积函数
            REAL(KIND=8) :: f, x
            f = SQRT(x + 2 * SQRT(x))
        END FUNCTION f
        FUNCTION g(x, y, z, u, v)
            REAL(KIND=8) :: g, x, y, z, u, v
            g = 5 - x**2 + y**2 - z**2 + u**3 - v**3
        END FUNCTION g
        
        SUBROUTINE integrate_1(n) !一维Monte Carlo积分子程序
            REAL(KIND=8) ,DIMENSION(10**n):: x
            REAL(KIND=8) :: ave, res
            INTEGER(KIND=4) :: i, n
            CALL Schrage(n, 54123654, 'rand.dat') !用16807产生器产生一定数目的均匀随机数
            OPEN (1, file='rand.dat')
            READ (1, *) x
            CLOSE (1)
            ave = f(5 * x(1)) !将0到1间随机数x转换为0到5间均匀分布随机数
            DO i = 2, 10**n
                ave = ave + (f(5 * x(i)) - ave) / i !为防止溢出调整了平均值求法
            END DO
            res = 5 * ave !积分计算结果
            print *, '1d', res
        END SUBROUTINE integrate_1

        SUBROUTINE integrate_2(n) !五维Monte Carlo积分子程序
            REAL(KIND=8) ,DIMENSION(10**(n + 1)) :: dat
            REAL(KIND=8) ,DIMENSION(10**n, 5) :: x
            REAL(KIND=8) :: ave, res
            INTEGER(KIND=4) :: i, j, n
            CALL Schrage(n + 1, 8455214, 'rand.dat') !用16807产生器产生一定数目的均匀随机数
            OPEN (1, file='rand.dat')
            READ (1, *) dat
            CLOSE (1)
            DO i = 1, 10**n
                DO j = 1, 5
                    x(i, j) = dat(5 * i + j) !每隔5个数从前面产生的随机数序列中取一个值,对应5个维度
                END DO
            END DO
            ave = g((7.0/10) * x(1, 1), (4.0/7) * x(1, 2),&
                  (9.0/10) * x(1, 3), 2.0 * x(1, 4),&
                  (13/11) * x(1, 5))
            !将0到1间均匀随机数变换为0到任意值之间均匀随机数
            DO i = 2, 10**n
                ave = ave + (g((7.0/10) * x(i, 1),&
                      (4.0/7) * x(i, 2),(9.0/10) * x(i, 3),&
                      2.0 * x(i,4),(13/11) * x(i,5))-ave)/i
            END DO
            res = (7.0/10) * (4.0/7) * (9.0/10) * 2 * (13.0/11) * ave
            print *, '5d', res
        END SUBROUTINE integrate_2
END MODULE integrate
\end{lstlisting}
\end{framed}

    \item \texttt{SUBROUTINE Schrage(P, z0, filename)}

        16807生成器子程序.(这是一个外部过程,调用了之前所写的子程序)

\begin{framed}
\begin{lstlisting}[language=Fortran]
SUBROUTINE Schrage(P, z0, filename) !Schrage随机数生成器子程序
    IMPLICIT NONE
    INTEGER :: N = 1, P
    INTEGER :: m = 2147483647, a = 16807, q = 127773, r = 2836, In(10**P), z0
    REAL(KIND=8) :: z(10**P)
    CHARACTER(LEN=8) :: filename
    In(1) = z0 !将传入值z0作为种子
    z(1) = REAL(In(1))/m
    DO N = 1, 10**P - 1
        In(N + 1) = a*MOD(In(N), q) - r*INT(In(N)/q)
        IF (In(N + 1) < 0) THEN !若值小于零,按Schrage方法加m
            In(N + 1) = In(N + 1) + m
        END IF
        z(N + 1) = REAL(In(N + 1))/m !得到第N+1个随机数
    END DO
    OPEN (1, file=trim(filename)) !每次运行子程序按照传入参数filename生成数据文件
    DO N = 1, 10**P !将随机数按行存入文件
        WRITE (1, *) z(N)
    END DO
    CLOSE (1)
END SUBROUTINE Schrage
\end{lstlisting}
\end{framed}

    
\end{itemize}

在主程序中,使用两个 \texttt{DO}循环结构,实现对不同抽样点数的两种积分.

\begin{framed}
\begin{lstlisting}[language=Fortran]
PROGRAM MAIN
    USE integrate
    INTEGER(KIND=4) :: i
    DO i = 2, 7
        CALL integrate_1(i)
    END DO
    DO i = 2, 7
        CALL integrate_2(i)
    END DO
END PROGRAM MAIN
\end{lstlisting}
\end{framed}

\section{计算结果}

\subsection{一重定积分}

将一重定积分程序计算结果列表展示如下:

\begin{table}[htb]
\centering
\begin{tabular}{|l|l|l|l|l|l|l|}
\hline
抽样点数$N$ & $10^2$   & $10^3$   & $10^4$   & $10^5$   & $10^6$   & $10^7$   \\ \hline
积分值  & 10.95739 & 11.35194 & 11.32440 & 11.30942 & 11.32128 & 11.31761 \\ \hline
\end{tabular}
\caption{一重定积分计算结果}
\end{table}

在Wolfram
Alpha中计算精确值为11.31796149(保留小数点后8位),于是我们可讨论有效数字位数.
\begin{itemize}
    \item $N=10^2$时,只有1位有效数字;
    \item $N=10^4$时,有2位有效数字;
    \item $N=10^6$时,有3位有效数字;
    \item $N=10^7$时,有4位有效数字.
\end{itemize}

从上面规律可以发现,要想提高1位数字精度,大概需要抽样数变为原来的100倍.

\subsection{多重定积分}

将五重定积分程序计算结果列表展示如下:

\begin{table}[htb]
\centering
\begin{tabular}{|l|l|l|l|l|l|l|}
\hline
抽样点数 & $10^2$  & $10^3$  & $10^4$  & $10^5$  & $10^6$  & $10^7$  \\ \hline
积分值  & 5.29003 & 5.51398 & 5.49307 & 5.46980 & 5.46718 & 5.46795 \\ \hline
\end{tabular}
\caption{五重定积分计算结果}
\end{table}

根据所得数据以及前一节中一重定积分有效数字位数的讨论,我们可以得到
\begin{itemize}
    \item $N=10^2$时,有1位有效数字;
    \item $N=10^4$时,有2位有效数字;
    \item $N=10^6$时,有3位有效数字;
    \item ($N=10^8$时,有4位有效数字.)
\end{itemize}

这与Monte Carlo积分理论上的误差相吻合.

\section{结论}

本作业中我们用Monte
Carlo方法计算了积分,显见要想使用MC方法达到较高的积分精度需要付出巨大的计算代价.
但是对于高维积分,其他数值积分方法变得难以操作,MC方法便体现出了其优越性.
\end{document}
